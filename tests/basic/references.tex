\documentclass{article}
\usepackage[utf8]{inputenc}
\usepackage[T1]{fontenc}

\title{Cross-References Test}
\author{Docker Test Suite}
\date{\today}

\begin{document}

\maketitle

\tableofcontents

\section{Introduction}
\label{sec:intro}
This document tests cross-referencing capabilities, which require multiple compilation passes.

\section{Sections and Labels}
\label{sec:sections}
This is Section~\ref{sec:sections}. We can also reference the Introduction in Section~\ref{sec:intro}.

\subsection{Subsection Example}
\label{subsec:example}
This subsection can be referenced as Subsection~\ref{subsec:example}.

\section{Figures and Tables}
\label{sec:figures}

\begin{table}[h]
\centering
\begin{tabular}{|l|r|}
\hline
Item & Value \\
\hline
Alpha & 1.0 \\
Beta & 2.5 \\
Gamma & 3.7 \\
\hline
\end{tabular}
\caption{Sample data table}
\label{tab:sample}
\end{table}

Table~\ref{tab:sample} shows some sample data that we reference from Section~\ref{sec:figures}.

\section{Equations}
\label{sec:equations}

The famous equation is shown in Equation~\ref{eq:einstein}:

\begin{equation}
E = mc^2
\label{eq:einstein}
\end{equation}

We can reference Equation~\ref{eq:einstein} from anywhere in the document.

\section{Conclusion}
This document successfully demonstrates cross-referencing between Section~\ref{sec:intro}, Section~\ref{sec:sections}, Section~\ref{sec:figures}, and Section~\ref{sec:equations}.

\end{document}