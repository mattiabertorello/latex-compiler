\documentclass{article}
\usepackage[utf8]{inputenc}
\usepackage[T1]{fontenc}
\usepackage{geometry}
\usepackage{amsmath}
\usepackage{amssymb}
\usepackage{amsthm}
\usepackage{graphicx}
\usepackage{booktabs}
\usepackage{caption}
\usepackage{subcaption}
\usepackage{hyperref}

\geometry{margin=1in}

\theoremstyle{definition}
\newtheorem{definition}{Definition}[section]
\newtheorem{theorem}{Theorem}[section]
\newtheorem{lemma}[theorem]{Lemma}
\newtheorem{corollary}[theorem]{Corollary}

\theoremstyle{remark}
\newtheorem{remark}{Remark}[section]

\title{Scientific Document Test}
\author{Docker Test Suite}
\date{\today}

\begin{document}

\maketitle

\begin{abstract}
This document tests scientific writing capabilities including theorems, proofs, complex mathematics, and structured scientific content formatting.
\end{abstract}

\section{Mathematical Foundations}

\begin{definition}[Metric Space]
A metric space is an ordered pair $(M, d)$ where $M$ is a set and $d$ is a metric on $M$, i.e., a function
\[
d: M \times M \to \mathbb{R}
\]
satisfying the following axioms for all $x, y, z \in M$:
\begin{enumerate}
    \item $d(x, y) \geq 0$ (non-negativity)
    \item $d(x, y) = 0 \iff x = y$ (identity of indiscernibles)
    \item $d(x, y) = d(y, x)$ (symmetry)
    \item $d(x, z) \leq d(x, y) + d(y, z)$ (triangle inequality)
\end{enumerate}
\end{definition}

\begin{theorem}[Banach Fixed Point Theorem]
\label{thm:banach}
Let $(X, d)$ be a non-empty complete metric space with a contraction mapping $T: X \to X$. Then $T$ admits a unique fixed-point $x^*$ in $X$ (i.e., $T(x^*) = x^*$).
\end{theorem}

\begin{proof}
Let $x_0 \in X$ be arbitrary and define the sequence $(x_n)$ by $x_{n+1} = T(x_n)$ for $n \geq 0$.

Since $T$ is a contraction, there exists $0 \leq q < 1$ such that
\[
d(T(x), T(y)) \leq q \cdot d(x, y)
\]
for all $x, y \in X$.

We can show that $(x_n)$ is a Cauchy sequence:
\[
d(x_{n+1}, x_n) = d(T(x_n), T(x_{n-1})) \leq q \cdot d(x_n, x_{n-1}) \leq q^n \cdot d(x_1, x_0)
\]

Since $X$ is complete, the sequence converges to some $x^* \in X$. By continuity of $T$:
\[
x^* = \lim_{n \to \infty} x_{n+1} = \lim_{n \to \infty} T(x_n) = T(x^*)
\]

Uniqueness follows from the contraction property.
\end{proof}

\begin{corollary}
Every contraction mapping on a complete metric space has exactly one fixed point.
\end{corollary}

\section{Complex Analysis}

\begin{lemma}[Cauchy-Riemann Equations]
Let $f(z) = u(x,y) + iv(x,y)$ be a complex function. If $f$ is differentiable at $z_0 = x_0 + iy_0$, then:
\begin{align}
\frac{\partial u}{\partial x} &= \frac{\partial v}{\partial y} \\
\frac{\partial u}{\partial y} &= -\frac{\partial v}{\partial x}
\end{align}
\end{lemma}

\section{Statistical Analysis}

The probability density function of a normal distribution is:
\[
f(x|\mu,\sigma^2) = \frac{1}{\sqrt{2\pi\sigma^2}} e^{-\frac{(x-\mu)^2}{2\sigma^2}}
\]

\begin{table}[h]
\centering
\caption{Statistical test results}
\begin{tabular}{lrrr}
\toprule
Test & Statistic & p-value & Significance \\
\midrule
t-test & 2.345 & 0.023 & * \\
ANOVA & 4.567 & 0.001 & ** \\
Chi-square & 12.34 & $< 0.001$ & *** \\
\bottomrule
\end{tabular}
\end{table}

\section{Matrix Operations}

Consider the eigenvalue problem:
\[
A\mathbf{v} = \lambda\mathbf{v}
\]

For a $2 \times 2$ matrix:
\[
A = \begin{pmatrix}
a & b \\
c & d
\end{pmatrix}
\]

The characteristic polynomial is:
\[
\det(A - \lambda I) = \lambda^2 - (a+d)\lambda + (ad-bc) = 0
\]

\begin{remark}
The trace of $A$ is $\text{tr}(A) = a + d$ and the determinant is $\det(A) = ad - bc$.
\end{remark}

\section{Conclusion}
This document successfully demonstrates scientific writing capabilities including:
\begin{itemize}
    \item Mathematical theorems and proofs
    \item Complex mathematical notation
    \item Scientific document structure
    \item Professional theorem environments
    \item Statistical and matrix notation
\end{itemize}

All scientific packages are functioning correctly as verified by Theorem~\ref{thm:banach} and related mathematical content.

\end{document}